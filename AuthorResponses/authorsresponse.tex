% Authors' response to PRA reviews of the manuscript formerly known as
% "Extreme Quantum Advantage when Simulating Classical Systems with
% 	Long-Range Interaction"
%
% jpc: 3/13/17, 3/21/17

\documentclass{article}
\usepackage{charter,graphicx,fancyhdr,geometry}

\geometry{paper=letterpaper,
  hmargin=1in,
  vmargin=1.5in,
  top = 0.25in,
  bottom = 1in,
  }

\pagestyle{fancy}
\fancyhf{}% Clear header/footer
\renewcommand{\headrulewidth}{0pt}% No header rule
\renewcommand{\footrulewidth}{1pt}% 1pt footer rule
\fancyfoot[C]{%
  \small
  \begin{tabular}{c}
    Complexity Sciences Center, Physics Department,
    University of California, Davis, CA 95616-5270\\
    http://csc.ucdavis.edu/
  \end{tabular}}
\fancyfoot[R]{\thepage}

\setlength{\parindent}{0pt}
\setlength{\parskip}{.5\baselineskip plus 2pt minus 2pt}

\usepackage{apacite}
\usepackage{epstopdf}
\usepackage{color}
\definecolor{UCDBlue}{rgb}{0.067, 0.278, 0.506}
\definecolor{UCDGold}{rgb}{0.855, 0.686, 0.024}

\usepackage{color}
\newcommand{\alert}[1]{\textbf{\textcolor{red}{#1}}}

\usepackage{amsmath}
%\usepackage{bordermatrix} % matrix with labeled row and col
%\usepackage{kbordermatrix} % matrix with labeled row and col
\usepackage{amscd}
\usepackage{amsmath}
\usepackage{amsfonts}
\usepackage{amssymb}
\usepackage{amsthm}
\usepackage{bigints}
\usepackage{braket}
\usepackage{xfrac}

\input{cmechabbrev}
\newcommand{\Abet}{\ProcessAlphabet}
\newcommand{\MS}{\MeasSymbol}
\newcommand{\MSs}{\MeasSymbols}
\newcommand{\ms}{\meassymbol}
\newcommand{\SSet}{\CausalStateSet}
\newcommand{\St}{\CausalState}
\newcommand{\st}{\causalstate}

\definecolor{reviewblack}{rgb}{0.5, 0.5, 0.5}
\definecolor{todored}{rgb}{0.8, 0.2, 0.2}
\definecolor{replyblue}{rgb}{0.1, 0.1, 0.8}
\definecolor{highlightorange}{rgb}{0.9, 0.5, 0.1}

\newcommand{\TODO}[1]{\textcolor{todored}{#1}}
\newcommand{\REVIEW}[1]{{ \it \textcolor{reviewblack}{#1}}}
\newcommand{\REPLY}[1]{\textcolor{UCDBlue}{#1}}
\newcommand{\HIGHLIGHT}[1]{\textbf{\textcolor{UCDGold}{#1}}}

\usepackage[colorlinks = True, urlcolor = UCDBlue]{hyperref}

\begin{document}
\includegraphics[height=3\baselineskip]{expanded_logo_cmyk_gold-blue}
\hspace{3.7in}
\includegraphics[height=5\baselineskip]{Logo_CSC}

\hrulefill

\begin{center}
Authors' response to referee's comments on\\
\vspace{0.1in}
\emph{The Markov Memory for Generating Rare Events}\\
by Aghamohammadi  and Crutchfield\\
\vspace{0.1in}
\emph{Physical Review E} EX11508\\
\end{center}

We are pleased the referees found the manuscript important, introducing
interesting ideas, attractive, and well written with intriguing examples.
Their feedback was quite useful. The reports raised several minor points that
the manuscript now addresses. We discuss these changes point-by-point.

\REPLY{Authors' responses in roman;} \REVIEW{referee comments in italics,
including points that we \HIGHLIGHT{highlight}}.

{\bf Editor}

\REVIEW{
The references in your manuscript do not conform to the style of this
journal, as can be found in any recent issue or from information at
http://journals.aps.org/ under the journal's specific home page.
}

\REPLY{
Removed article names from the bibliography.
}

\REVIEW{
In reviewing the figures of your paper, we note that the following
changes would be needed in order for your figures to conform to the
style of the Physical Review.� Please check all figures for the
following problems and make appropriate changes in the text of the
paper itself wherever needed for consistency.
Please provide a PostScript or EPS figure file that does not require
commands in the text source file to include axis labeling or other
such content. Figure files should be self-contained. We will also
need a revised text source file.
}

\REPLY{
{Change PDF figures to EPS.}
}

\REVIEW{
Figure sub-labels should be printed on the figures. The preferred form
is lowercase letters in parentheses: (a), (b), etc.
}

\REPLY{Corrected.}

{\bf Reviewer \#1}
%%%%%%%%%%%%%%%%%%%%%%%%%%%%%%%%%%%%%%%%%%%%%%

\REVIEW{
In this paper, for a given process, the authors have devised a new
process which generates rare events as the typical events for any
chosen rare class by means of finite-state hidden Markov models. Then,
by obtaining the complexity measure of a typical process, which has
been studied before and is based on a process' \eM, they
calculated the required memory for generating rare events of processes
generated by a unique hidden Markov model. By exploiting three
examples it is shown that the memory cost vary in different classes of
rare events.
\HIGHLIGHT{The importance of the proposed method could be considerable in
statistical model selection since previously only the complexity of
the typical processes was considered however it also relies on
required memory for rare events.
Furthermore, this method extends the complexity measure over the full
memory spectrum to generate fluctuations, thus gives one the chance to
study the rare behaviors of many stochastic systems and the
difficulty to simulate them.
In summary, I believe that this paper brings interesting ideas and
results in order to be published in the in PRE}
}

\REPLY{We respond to the questions in the following.}

\REVIEW{
Talking about the HMMs from the beginning of the paper, the author has provided its details in section IV. Placing this explanation beforehand will improve the paper's cohesion.}

\REPLY{
Problem background and explanation has been moved to beginning. More context
and details were also added.}

\REVIEW{
Fig.~$4$ top right better to be labeled similar to other graphs $(p|x)$.
}

\REPLY{Corrected.}

\REVIEW{
In Fig.~$4$ the legend of $\Cmu$ graph should be ``Intermittent Periodic
Process'' (IPP) as it is throughout the paper and it would be more
clear if the limit of $\beta \to -\infty (P=1)$ and $\beta \to \infty (P=0)$ were depicted as
insets, such as Fig.~$5$.
}

\REPLY{Corrected.} 

\REVIEW{
In describing Fig.~$4$ the ``red'' dashed line has been mentioned wrongly
as ``green''. In section IV, ``HHMs'' must be ``HMMs''.
}

\REPLY{Corrected.}

{\bf Reviewer \#2}

\REVIEW{
\HIGHLIGHT{This is an interesting paper discussing Markov chain models with hidden states,
or Hidden Markov Models, that generates bit streams that from the external world
appear to be non-Markovian.}� The second author has done much work in this area,
and although it has close relations to theoretical issues in computer science,
much of the previous work has been published in physics journals including
Physical Review E.� \HIGHLIGHT{So it appears to be appropriate for this journal.}
The authors introduce various hidden Markov models, including the Two-Biased
Coin Process, the Intermittent Periodic Process, the Even Process, and \HIGHLIGHT{these
models are intriguing} and not (I believe) commonly studied in statistical
physics.� The authors develop a way to construct a new process that can generate
desired rare events from the original process.� (More specific examples of rare
event sequences might be useful).�\HIGHLIGHT{The paper is also attractively presented with
nice color figures and clear explanations.}� A proof is relegated to an appendix.
}

\REPLY{
Thank you. We respond to the concerns in the following.
}

\REVIEW{
If it were possible to relate this paper to things studied more commonly by the
readers of this journal, that would be useful.� For example, when I hear rare
events I think of Monte-Carlo simulations were certain transition states occur
very rarely but are important to understand a phase transition, for example. 
Are any of these rare events related to problems like that?
}

\REPLY{
Helpful suggestion. Thank you. The relation between Monte Carlo algorithms and finite-state machine based algorithms has been added.
}

\REVIEW{
Just one correction: ``it's'' on p.~$2$ should be ``its''.
}

\REPLY{Corrected.}

\end{document}

