\PassOptionsToPackage{table}{xcolor}
\documentclass[final,serif,aspectratio=1610]{beamer}

\usepackage{dynlearn}

%%%%%%%%%%%%%%%%%%%%%%%%%%%%%%%%%%%%%%%%%%%%%%%%%%%%%%%%%%%%%%%%%%%%%%%%%%%%%%
% Beamer configuration

\usetheme{Darmstadt}

\beamertemplatenavigationsymbolsempty

\setbeamercolor{structure}{fg=ryan}

\makeatletter
  \let\frametextheight\beamer@frametextheight
\makeatother

% \AtBeginSection[]
% {
%   \begin{frame}
%     \tableofcontents[
%       currentsection,
%       currentsubsection,
%       hideothersubsections,
%       sectionstyle=show/shaded,
%       % subsectionstyle=show/hide,
%     ]
%   \end{frame}
% }

\tcbset{colback=ryan!5!white, colframe=ryan}

%%%%%%%%%%%%%%%%%%%%%%%%%%%%%%%%%%%%%%%%%%%%%%%%%%%%%%%%%%%%%%%%%%%%%%%%%%%%%%%%
% Definitions specific to this talk

\input{project_defs}

\addbibresource{chaos.bib}
\addbibresource{project.bib}

%%%%%%%%%%%%%%%%%%%%%%%%%%%%%%%%%%%%%%%%%%%%%%%%%%%%%%%%%%%%%%%%%%%%%%%%%%%%%%
% Talk info

\title{%
  \Huge Secret Key Agreement and Unique Information
}

\author[RGJ]{\Large\textbf{Ryan G. James}}

\institute[UCD]{%
  \csclogo
}

\DTMsavedate{date}{2018-10-20}

\date{%
  % \includegraphics[height=1.5cm]{images/ccs_logo}
  APS Far West Section \\
  \DTMusedate{date}
}

%%%%%%%%%%%%%%%%%%%%%%%%%%%%%%%%%%%%%%%%%%%%%%%%%%%%%%%%%%%%%%%%%%%%%%%%%%%%%%
% The Talk

\begin{document}

\begin{frame}
  \titlepage
\end{frame}

\section{Introduction}

\begin{frame}
  \frametitle{}
  \framesubtitle{}
  \begin{columns}
    \begin{column}{1.0\textwidth}

    \end{column}
  \end{columns}
\end{frame}

\section{Partial Information Decomposition}

\subsection{Definitions}

\begin{frame}
  \frametitle{Partial Information Decomposition~\cite{williams2010nonnegative}}
  \framesubtitle{Sources-Target Decomposition}
  \begin{columns}
    \begin{column}{1.0\textwidth}
      \Large

      Given \emph{sources} \source{0}, \source{1}, and a \emph{target} \target, we seek to decompose the information that the sources have about the target:
      \begin{align*}
        \I{\source{0}\source{1} : \target} =
          & \phantom{+}~\Ipart{\source{0}\sep\source{1}}
          & \textrm{\emph{redundant}} \\
          & + \Ipart[\target \setminus \source{1}]{\source{0}}
          & \textrm{\emph{unique with \source{0}}} \\
          & + \Ipart[\target \setminus \source{0}]{\source{1}}
          & \textrm{\emph{unique with \source{1}}} \\
          & + \Ipart{\source{0}\source{1}}
          & \textrm{\emph{synergistic}}
        \label{eq:decomp}
      \end{align*}
    \end{column}
  \end{columns}
\end{frame}

\begin{frame}
  \frametitle{Partial Information Decomposition~\cite{williams2010nonnegative}}
  \framesubtitle{Source-Target Decomposition}
  \begin{columns}
    \begin{column}{1.0\textwidth}
      \Large
      Furthermore, each source shares information with the target:
      \begin{align*}
        \I{\source{0} : \target} =
          & \phantom{+}~\Ipart{\source{0}\sep\source{1}}
          & \textrm{\emph{redundant}} \\
          & + \Ipart[\target \setminus \source{1}]{\source{0}}
          & \textrm{\emph{unique with \source{0}}}
      \end{align*}
      \begin{align*}
        \I{\source{1} : \target} =
          & \phantom{+}~\Ipart{\source{0}\sep\source{1}}
          & \textrm{\emph{redundant}} \\
          & + \Ipart[\target \setminus \source{0}]{\source{1}}
          & \textrm{\emph{unique with \source{1}}}
      \end{align*}
      \pause When quantifying unique informations, we require consistency:
      \begin{align*}
        \Ipart[\target \setminus \source{1}]{\source{0}} + \I{\source{1} : \target} = \Ipart[\target \setminus \source{0}]{\source{1}} + \I{\source{0} : \target}
      \end{align*}
    \end{column}
  \end{columns}
\end{frame}

\subsection{Implementations}

\begin{frame}
  \frametitle{Extant Measures}
  \framesubtitle{Oh my God, it's full of measures!}
  \begin{columns}
    \begin{column}{0.5\textwidth}
      \Large
      \begin{itemize}
        \item \Imin[]{}~\cite{williams2010nonnegative}
        \item \Iproj[]{}~\cite{harder2013bivariate}
        \item \Ibroja[]{}~\cite{bertschinger2014quantifying,banerjee2017computing}
        \item \Immi[]{}~\cite{bertschinger2013shared}
        \item \Iwedge[]{}~\cite{griffith2014intersection}
        \item \Irr[]{}~\cite{goodwell2017temporal}
        \item \Iccs[]{}~\cite{ince2017measuring}
      \end{itemize}
    \end{column}
    \begin{column}{0.5\textwidth}
      \Large
      \begin{itemize}
        \item \IRA[]{}~\cite{zwick2004overview}
        \item \Ipm[]{}~\cite{finn2018pointwise}
        \item \Idep[]{}~\cite{james2017unique,kay2018exact}
        \item \Irav[]{}~\cite{dit}
        \item \IUIi[]{}~\cite{banerjee2018unique}
        \item \IUIo[]{}~\cite{banerjee2018unique}
      \end{itemize}
    \end{column}
  \end{columns}
\end{frame}

\section{Secret Key Agreement Rate}

\subsection{Setup}

\begin{frame}
  \frametitle{Cryptographic Secrecy~\cite{maurer1993secret}}
  \framesubtitle{I am a cipher, a cipher wrapped in an enigma, smothered in secret sauce}
  \large
  Alice, $A$, and Bob, $B$, wish to agree upon a \emph{secret key}, $K$, despite Eve, $E$, eavesdropping:

  \vspace{1cm}

  \begin{columns}[t]
    \begin{column}{0.5\textwidth}
      \begin{description}[Communication:]
        \item[given:] $n$ samples of $p(a, b, e)$
        \item[Alice receives:] $A^n$
        \item[Bob receives:] $B^n$
        \item[Eve receives:] $E^n$
        \item[Communication:] $C$
      \end{description}
    \end{column}
    \begin{column}{0.5\textwidth}
      \onslide<2->{
        $S$ is the largest $R$ such that:
        \begin{align*}
          K_A &= f(A^n, C) \\
          K_B &= g(B^n, C) \\
          p(K_A = K_B = K) &\geq 1 - \epsilon \\
          \I{K : C E^n} &\leq \epsilon \\
          \frac{1}{n} \H{K} &\geq R - \epsilon
        \end{align*}
      }
    \end{column}
  \end{columns}
\end{frame}

\subsection{Punchline}

\begin{frame}
  \frametitle{Types of Secret Key Agreement Rate}
  \framesubtitle{You talkin' to me?}
  \begin{columns}
    \begin{column}{1.0\textwidth}
      \LARGE
      \begin{center}
        \begin{description}[\SKARtwo{A}{B}{E}:]
          \item[\SKARzero{A}{B}{E}:] no communication
          \item[\SKARoner{A}{B}{E}:] only Alice communicates
          \item[\SKARonel{A}{B}{E}:] only Bob communicates
          \item[\SKARtwo{A}{B}{E}:] both Alice and Bob communicate
        \end{description}
      \end{center}
    \end{column}
  \end{columns}
\end{frame}

\subsection{Quantifying}

\begin{frame}
  \frametitle{Computing Secret Key Agreement Rates}
  \framesubtitle{No Communication}
  \begin{columns}
    \begin{column}{1.0\textwidth}
      \Large
      In the case of neither Alice nor Bob communicating~\cite{chitambar2018conditional}:
      \begin{align*}
        \SKARzero{A}{B}{E} = \H{A \meet B | E}
      \end{align*}

      \begin{itemize}
        \item generally zero
        \item ``brittle''
      \end{itemize}
    \end{column}
  \end{columns}
\end{frame}

\begin{frame}
  \frametitle{Computing Secret Key Agreement Rates}
  \framesubtitle{One-Way Communication}
  \begin{columns}
    \begin{column}{1.0\textwidth}
      \Large
      When only Alice can communicate~\cite{ahlswede1993common}:
      \begin{align*}
        \SKARone{A}{B}{E} = \max_{C \markov K \markov A \markov BE} \I{B : K | C} - \I{E : K | C}
      \end{align*}

      \begin{itemize}
        \item $|K| \leq |A|$
        \item $|C| \leq |A|^2$
      \end{itemize}
    \end{column}
  \end{columns}
\end{frame}

\begin{frame}
  \frametitle{Computing Secret Key Agreement Rates}
  \framesubtitle{Two-Way Communication}
  \begin{columns}
    \begin{column}{1.0\textwidth}
      \Large
      When both Alice and Bob can communicate:
      \begin{align*}
        \SKARtwo{A}{B}{E} = \textrm{\textinterrobang}
      \end{align*}

      However these simple bounds hold:
      \begin{align*}
        &\max \left\{ \SKARone{A}{B}{E}, \SKARone{B}{A}{E} \right\} \\
        &\quad\quad \leq \SKARtwo{A}{B}{E} \leq \\
        &\quad\quad\quad\quad \I{A : B \downarrow Z} = \min_{p(\overline{z}|z)} \I{A : B | \overline{Z}}
      \end{align*}
      and even tighter ones exist~\cite{gohari2017coding}.
    \end{column}
  \end{columns}
\end{frame}

\section{Combined}

\subsection{Intuitions}

\begin{frame}
  \frametitle{\textsc{Pointwise Unique}}
  \framesubtitle{Properties}
  \begin{columns}
    \begin{column}{0.33\textwidth}
      \centering
      \begin{tikzpicture}
        \node at (0, 0) {\pntunqdist};
      \end{tikzpicture}
    \end{column}
    \begin{column}{0.66\textwidth}
    \end{column}
  \end{columns}
  \begin{align*}
    \I{\source{0} : \target} = \SI{1/2}{\bit} \quad \quad \quad \quad \I{\source{1} : \target} = \SI{1/2}{\bit} \\
    \Ipart[\target \setminus \source{1}]{\source{0}} + \I{\source{1} : \target} = \Ipart[\target \setminus \source{0}]{\source{1}} + \I{\source{0} : \target}
  \end{align*}
\end{frame}

\begin{frame}
  \frametitle{\textsc{Pointwise Unique}}
  \framesubtitle{No Communication}
  \begin{columns}
    \begin{column}{0.33\textwidth}
      \centering
      \begin{tikzpicture}
        \node at (0, 0) {\pntunqdist};
      \end{tikzpicture}
    \end{column}
    \begin{column}{0.66\textwidth}
      \centering
      \begin{tabular}{lr}
        \multicolumn{2}{c}{\SKARzero{\source{i}}{\target}{\source{j}}} \\
        \toprule
        \Ipart{\source{0}\source{1}}                     & \SI{1/2}{\bit} \\
        \Ipart[\target \setminus \source{1}]{\source{0}} & \SI{0}{\bit}   \\
        \Ipart[\target \setminus \source{0}]{\source{1}} & \SI{0}{\bit}   \\
        \Ipart{\source{0}\sep\source{1}}                 & \SI{1/2}{\bit} \\
        \bottomrule
      \end{tabular}
    \end{column}
  \end{columns}
  \begin{align*}
    \I{\source{0} : \target} = \SI{1/2}{\bit} \quad \quad \quad \quad \I{\source{1} : \target} = \SI{1/2}{\bit} \\
    \Ipart[\target \setminus \source{1}]{\source{0}} + \I{\source{1} : \target} = \Ipart[\target \setminus \source{0}]{\source{1}} + \I{\source{0} : \target}
  \end{align*}
\end{frame}

\begin{frame}
  \frametitle{\textsc{Pointwise Unique}}
  \framesubtitle{One-Way Communication: Camel}
  \begin{columns}
    \begin{column}{0.33\textwidth}
      \centering
      \begin{tikzpicture}
        \node at (0, 0) {\pntunqdist};
      \end{tikzpicture}
    \end{column}
    \begin{column}{0.66\textwidth}
      \centering
      \begin{tabular}{lr}
        \multicolumn{2}{c}{\SKARone{\source{i}}{\target}{\source{j}}} \\
        \toprule
        \Ipart{\source{0}\source{1}}                     & \SI{0}{\bit}   \\
        \Ipart[\target \setminus \source{1}]{\source{0}} & \SI{1/2}{\bit} \\
        \Ipart[\target \setminus \source{0}]{\source{1}} & \SI{1/2}{\bit} \\
        \Ipart{\source{0}\sep\source{1}}                 & \SI{0}{\bit}   \\
        \bottomrule
      \end{tabular}
    \end{column}
  \end{columns}
  \begin{align*}
    \I{\source{0} : \target} = \SI{1/2}{\bit} \quad \quad \quad \quad \I{\source{1} : \target} = \SI{1/2}{\bit} \\
    \Ipart[\target \setminus \source{1}]{\source{0}} + \I{\source{1} : \target} = \Ipart[\target \setminus \source{0}]{\source{1}} + \I{\source{0} : \target}
  \end{align*}
\end{frame}

\begin{frame}
  \frametitle{\textsc{Pointwise Unique}}
  \framesubtitle{One-Way Communication: Elephant}
  \begin{columns}
    \begin{column}{0.33\textwidth}
      \centering
      \begin{tikzpicture}
        \node at (0, 0) {\pntunqdist};
      \end{tikzpicture}
    \end{column}
    \begin{column}{0.66\textwidth}
      \centering
      \begin{tabular}{lr}
        \multicolumn{2}{c}{\SKARonel{\source{i}}{\target}{\source{j}}} \\
        \toprule
        \Ipart{\source{0}\source{1}}                     & \SI{1/2}{\bit} \\
        \Ipart[\target \setminus \source{1}]{\source{0}} & \SI{0}{\bit}   \\
        \Ipart[\target \setminus \source{0}]{\source{1}} & \SI{0}{\bit}   \\
        \Ipart{\source{0}\sep\source{1}}                 & \SI{1/2}{\bit} \\
        \bottomrule
      \end{tabular}
    \end{column}
  \end{columns}
  \begin{align*}
    \I{\source{0} : \target} = \SI{1/2}{\bit} \quad \quad \quad \quad \I{\source{1} : \target} = \SI{1/2}{\bit} \\
    \Ipart[\target \setminus \source{1}]{\source{0}} + \I{\source{1} : \target} = \Ipart[\target \setminus \source{0}]{\source{1}} + \I{\source{0} : \target}
  \end{align*}
\end{frame}

\begin{frame}
  \frametitle{\textsc{Pointwise Unique}}
  \framesubtitle{Two-Way Communication}
  \begin{columns}
    \begin{column}{0.33\textwidth}
      \centering
      \begin{tikzpicture}
        \node at (0, 0) {\pntunqdist};
      \end{tikzpicture}
    \end{column}
    \begin{column}{0.66\textwidth}
      \centering
      \begin{tabular}{clr}
        \multicolumn{2}{c}{\SKARtwo{\source{i}}{\target}{\source{j}}} \\
        \toprule
        \Ipart{\source{0}\source{1}}                     & \SI{0}{\bit}   \\
        \Ipart[\target \setminus \source{1}]{\source{0}} & \SI{1/2}{\bit} \\
        \Ipart[\target \setminus \source{0}]{\source{1}} & \SI{1/2}{\bit} \\
        \Ipart{\source{0}\sep\source{1}}                 & \SI{0}{\bit}   \\
        \bottomrule
      \end{tabular}
    \end{column}
  \end{columns}
  \begin{align*}
    \I{\source{0} : \target} = \SI{1/2}{\bit} \quad \quad \quad \quad \I{\source{1} : \target} = \SI{1/2}{\bit} \\
    \Ipart[\target \setminus \source{1}]{\source{0}} + \I{\source{1} : \target} = \Ipart[\target \setminus \source{0}]{\source{1}} + \I{\source{0} : \target}
  \end{align*}
\end{frame}

\subsection{Problems}

\begin{frame}
  \frametitle{\textsc{Problem}}
  \framesubtitle{}
  \begin{columns}
    \begin{column}{0.33\textwidth}
      \centering
      \begin{tikzpicture}
        \node at (0, 0) {\probdist};
      \end{tikzpicture}
    \end{column}
    \begin{column}{0.66\textwidth}
    \end{column}
  \end{columns}
  \begin{align*}
    \I{\source{0} : \target} = \SI{0.3113}{\bit} \quad \quad \quad \quad \I{\source{1} : \target} = \SI{1/2}{\bit} \\
    \Ipart[\target \setminus \source{1}]{\source{0}} + \I{\source{1} : \target} = \Ipart[\target \setminus \source{0}]{\source{1}} + \I{\source{0} : \target}
  \end{align*}
\end{frame}

\begin{frame}
  \frametitle{\textsc{Problem}}
  \framesubtitle{No Communication}
  \begin{columns}
    \begin{column}{0.33\textwidth}
      \centering
      \begin{tikzpicture}
        \node at (0, 0) {\probdist};
      \end{tikzpicture}
    \end{column}
    \begin{column}{0.66\textwidth}
      \centering
      \begin{tabular}{lr}
        \multicolumn{2}{c}{\SKARzero{\source{i}}{\target}{\source{j}}} \\
        \toprule
        \Ipart{\source{0}\source{1}}                     & \xmark       \\
        \Ipart[\target \setminus \source{1}]{\source{0}} & \SI{0}{\bit} \\
        \Ipart[\target \setminus \source{0}]{\source{1}} & \SI{0}{\bit} \\
        \Ipart{\source{0}\sep\source{1}}                 & \xmark       \\
        \bottomrule
      \end{tabular}
    \end{column}
  \end{columns}
  \begin{align*}
    \I{\source{0} : \target} = \SI{0.3113}{\bit} \quad \quad \quad \quad \I{\source{1} : \target} = \SI{1/2}{\bit} \\
    \Ipart[\target \setminus \source{1}]{\source{0}} + \I{\source{1} : \target} = \Ipart[\target \setminus \source{0}]{\source{1}} + \I{\source{0} : \target}
  \end{align*}
\end{frame}

\begin{frame}
  \frametitle{\textsc{Problem}}
  \framesubtitle{One-Way Communication: Camel}
  \begin{columns}
    \begin{column}{0.33\textwidth}
      \centering
      \begin{tikzpicture}
        \node at (0, 0) {\probdist};
      \end{tikzpicture}
    \end{column}
    \begin{column}{0.66\textwidth}
      \centering
      \begin{tabular}{lr}
        \multicolumn{2}{c}{\SKARone{\source{i}}{\target}{\source{j}}} \\
        \toprule
        \Ipart{\source{0}\source{1}}                     & \xmark         \\
        \Ipart[\target \setminus \source{1}]{\source{0}} & \SI{0}{\bit}   \\
        \Ipart[\target \setminus \source{0}]{\source{1}} & \SI{1/2}{\bit} \\
        \Ipart{\source{0}\sep\source{1}}                 & \xmark         \\
        \bottomrule
      \end{tabular}
    \end{column}
  \end{columns}
  \begin{align*}
    \I{\source{0} : \target} = \SI{0.3113}{\bit} \quad \quad \quad \quad \I{\source{1} : \target} = \SI{1/2}{\bit} \\
    \Ipart[\target \setminus \source{1}]{\source{0}} + \I{\source{1} : \target} = \Ipart[\target \setminus \source{0}]{\source{1}} + \I{\source{0} : \target}
  \end{align*}
\end{frame}

\begin{frame}
  \frametitle{\textsc{Problem}}
  \framesubtitle{One-Way Communication: Elephant}
  \begin{columns}
    \begin{column}{0.33\textwidth}
      \centering
      \begin{tikzpicture}
        \node at (0, 0) {\probdist};
      \end{tikzpicture}
    \end{column}
    \begin{column}{0.66\textwidth}
      \centering
      \begin{tabular}{lr}
        \multicolumn{2}{c}{\SKARonel{\source{i}}{\target}{\source{j}}} \\
        \toprule
        \Ipart{\source{0}\source{1}}                     & \SI{1/2}{\bit} \\
        \Ipart[\target \setminus \source{1}]{\source{0}} & \SI{0}{\bit}   \\
        \Ipart[\target \setminus \source{0}]{\source{1}} & \SI{0.1887}{\bit}   \\
        \Ipart{\source{0}\sep\source{1}}                 & \SI{0.3113}{\bit} \\
        \bottomrule
      \end{tabular}
    \end{column}
  \end{columns}
  \begin{align*}
    \I{\source{0} : \target} = \SI{0.3113}{\bit} \quad \quad \quad \quad \I{\source{1} : \target} = \SI{1/2}{\bit} \\
    \Ipart[\target \setminus \source{1}]{\source{0}} + \I{\source{1} : \target} = \Ipart[\target \setminus \source{0}]{\source{1}} + \I{\source{0} : \target}
  \end{align*}
\end{frame}

\begin{frame}
  \frametitle{\textsc{Problem}}
  \framesubtitle{Two-Way Communication}
  \begin{columns}
    \begin{column}{0.33\textwidth}
      \centering
      \begin{tikzpicture}
        \node at (0, 0) {\probdist};
      \end{tikzpicture}
    \end{column}
    \begin{column}{0.66\textwidth}
      \centering
      \begin{tabular}{lr}
        \multicolumn{2}{c}{\SKARtwo{\source{i}}{\target}{\source{j}}} \\
        \toprule
        \Ipart{\source{0}\source{1}}                     & \xmark                   \\
        \Ipart[\target \setminus \source{1}]{\source{0}} & $\leq \SI{0.1887}{\bit}$ \\
        \Ipart[\target \setminus \source{0}]{\source{1}} & \SI{1/2}{\bit}           \\
        \Ipart{\source{0}\sep\source{1}}                 & \xmark                   \\
        \bottomrule
      \end{tabular}
    \end{column}
  \end{columns}
  \begin{align*}
    \I{\source{0} : \target} = \SI{0.3113}{\bit} \quad \quad \quad \quad \I{\source{1} : \target} = \SI{1/2}{\bit} \\
    \Ipart[\target \setminus \source{1}]{\source{0}} + \I{\source{1} : \target} = \Ipart[\target \setminus \source{0}]{\source{1}} + \I{\source{0} : \target}
  \end{align*}
\end{frame}

\section{Conclusion}

\subsection{Thank You!}

\begin{frame}
  \begin{columns}
    \begin{column}{0.6\textwidth}
      \begin{tcolorbox}[title=Summary]
        \setlength{\leftmargini}{0.1cm}
        \setlength{\leftmarginii}{0.25cm}
        \large
        \begin{itemize}
          \item Most proposed PIDs do not have an operational meaning
          \item Utilizing secret key agreement rate as unique information largely doesn't work:
          \begin{itemize}
            \item \SKARzero{\source{i}}{\target}{\source{j}} is inconsistent
            \item \SKARoner{\source{i}}{\target}{\source{j}} is inconsistent
            \item \SKARtwo{\source{i}}{\target}{\source{j}} is inconsistent
            \item \SKARonel{\source{i}}{\target}{\source{j}} mandates elephant thinking
          \end{itemize}
        \end{itemize}
      \end{tcolorbox}
    \end{column}
    \begin{column}{0.4\textwidth}
      \begin{tcolorbox}[title=Collaborators]
        \setlength{\leftmargini}{0.25cm}
        \begin{itemize}
          \item Jeff Emenheiser
          \item Jim Crutchfield
          \item Daniel Feldspar
        \end{itemize}
      \end{tcolorbox}
      \vspace{0.11cm}
      \begin{tcolorbox}[title=Calculations]
        All calculations seen here were computed using the \texttt{dit} Python package:

        \vspace{0.25cm}
        \footnotesize
        \fullcite{dit}
      \end{tcolorbox}
    \end{column}
  \end{columns}
\end{frame}

\begin{frame}[allowframebreaks]
  \frametitle{References}
  \printbibliography[heading=none]
\end{frame}

\appendix

\section{Necessary Communication}

\begin{frame}
  \frametitle{}
  \framesubtitle{}
  \begin{columns}
    \begin{column}{1.0\textwidth}
      \Large

      \begin{align*}
        \I{\source{0} | \target | \source{1}} =
        &\Ipart[\target \setminus \source{1}]{\source{0}}
        & \textrm{\emph{unique with \source{0}}} \\
        & + \Ipart{\source{0}\source{1}}
        & \textrm{\emph{synergistic}}
      \end{align*}
      \begin{align*}
        \I{\source{1} | \target | \source{0}} =
        &\Ipart[\target \setminus \source{0}]{\source{1}}
        & \textrm{\emph{unique with \source{1}}} \\
        & + \Ipart{\source{0}\source{1}}
        & \textrm{\emph{synergistic}}
      \end{align*}
    \end{column}
  \end{columns}
\end{frame}

\begin{frame}
  \frametitle{}
  \framesubtitle{}
  \begin{columns}
    \begin{column}{0.66\textwidth}
      \centering
      \begin{tikzpicture}
        \node at (0, 0)   [anchor=north] {\gbdist};
        \node at (3.5, 0) [anchor=north] {\twowaydist};

        \node at (0, -3.75) {$BEC(p)$};
        \node (a) at (-1, -4.5) {0};
        \node (b) at (-1, -6.5) {1};
        \node (c) at (1, -4.5)  {0};
        \node (d) at (1, -5.5)  {$\epsilon$};
        \node (e) at (1, -6.5)  {1};

        \draw [->, >=stealth] (a) -- node[above] {$\pbar$} (c);
        \draw [->, >=stealth] (a) -- node[below] {$p$}     (d);
        \draw [->, >=stealth] (b) -- node[above] {$p$}     (d);
        \draw [->, >=stealth] (b) -- node[below] {$\pbar$} (e);
      \end{tikzpicture}
    \end{column}
    \begin{column}{0.33\textwidth}
      \begin{align*}
        \SKARtwo{\source{0}}{\target}{\source{1}} = \I{\source{0} : \target | \source{1}}
      \end{align*}
      \begin{align*}
        I_3 > 0
      \end{align*}
    \end{column}
  \end{columns}
\end{frame}

\begin{frame}
  \frametitle{}
  \framesubtitle{}
  \begin{columns}
    \begin{column}{1.0\textwidth}
      here lies the dependency decomposition to explain how $I_3$ is part of the cmi.
    \end{column}
  \end{columns}
\end{frame}

\section{\relax \Ibroja[]{}}

\begin{frame}
  \frametitle{}
  \framesubtitle{}
  \begin{columns}
    \begin{column}{1.0\textwidth}
      \begin{align*}
        Q = \left\{ q(\source{0}, \source{1}, \target) : \forall i,~ q(\source{i}, \target) = p(\source{i}, \target) \right\}
      \end{align*}

    \end{column}
    \begin{column}{1.0\textwidth}
      \begin{tikzpicture}
        \node at (0, 0)   [anchor=north] {\pntunqa};
        \node at (3.5, 0) [anchor=north] {\pntunqb};
      \end{tikzpicture}
    \end{column}
  \end{columns}
\end{frame}

\end{document}
