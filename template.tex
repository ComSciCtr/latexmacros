\documentclass[prl,twocolumn,showpacs,groupaddress,preprintnumbers,floatfix]{revtex4-1}

\usepackage{dynlearn}

\begin{document}

\def\ourTitle{%
  Title goes here.
}

\def\ourAbstract{%
  Abstract goes here.
}

\def\ourKeywords{%
  stochastic process, hidden Markov model, \texorpdfstring{\eM}{epsilon-machine}, causal states, mutual information.
}

\hypersetup{
  pdfauthor={James P. Crutchfield},
  pdftitle={\ourTitle},
  pdfsubject={\ourAbstract},
  pdfkeywords={\ourKeywords},
  pdfproducer={},
  pdfcreator={}
}

\author{Ryan G. James}
\email{rgjames@ucdavis.edu}
\affiliation{Complexity Sciences Center and Physics Department,
University of California at Davis, One Shields Avenue, Davis, CA 95616}

% \author{John Mahoney}
% \email{jmahony@ucdavis.edu}
% \affiliation{Complexity Sciences Center and Physics Department,
% University of California at Davis, One Shields Avenue, Davis, CA 95616}

\author{James P. Crutchfield}
\email{chaos@ucdavis.edu}
\affiliation{Complexity Sciences Center and Physics Department,
University of California at Davis, One Shields Avenue, Davis, CA 95616}
% \affiliation{Santa Fe Institute, 1399 Hyde Park Road, Santa Fe, NM 87501}

\date{\today}
\bibliographystyle{unsrt}


%%%%%%%%%%%%%%%%%%%%%%%%%%%%%%%%%%%%%%%%%%%%%%%%%%%%%%%%%%%%%%%%%%%%%%%%%%%%%%%
% The paper content

\title{\ourTitle}

\begin{abstract}

\ourAbstract

\vspace{0.1in}
\noindent
{\bf Keywords}: \ourKeywords

\end{abstract}

\pacs{
05.45.-a  %  Nonlinear dynamics and nonlinear dynamical systems
89.75.Kd  %  Complex Systems: Patterns
89.70.+c  %  Information science
05.45.Tp  %  Time series analysis
%02.50.Ey  %  Stochastic processes
%02.50.-r  %  Probability theory, stochastic processes, and statistics
%02.50.Ga  %  Markov processes
%05.20.-y  %  Classical statistical mechanics
}

\preprint{\sfiwp{13-11-XXX}}
\preprint{\arxiv{1312.XXXX}}

\title{\ourTitle}
\date{\today}
\maketitle
\tableofcontents

\setstretch{1.1}



\section{Introduction}
\label{sec:introduction}

This is an introduction. See \cref{fig:even_process} for an example of an \eM. See \cref{fig:feedback_xor_channel} for an example of an \eT.

\begin{figure}
  \centering
  \begin{tikzpicture}[style=vaucanson,
                      bend angle=15,
                      scale=1,
                      every node/.style={transform shape}]
    \node [state] (A)              {A};
    \node [state] (B) [right of=A] {B};

    \path (A) edge [loop left] node {$\Edge{0}{\half}$} (A)
          (A) edge [bend left] node {$\Edge{1}{\half}$} (B)
          (B) edge [bend left] node {$\Edge{1}{1}$}     (A);
    % added for symmetry
    \path (B) edge [loop right, draw=none] node {} (B);
  \end{tikzpicture}
  \caption{The \eM for the Even process.}
  \label{fig:even_process}
\end{figure}

\begin{figure}
  \centering
  \begin{tikzpicture}[style=vaucanson,
                      bend angle=15,
                      scale=1,
                      every node/.style={transform shape}]
    \node [state] (A)              {A};
    \node [state] (B) [right of=A] {B};

    \path (A) edge [loop left]  node {$\TEdge{0}{0}{1}$} (A)
          (A) edge [bend left]  node {$\TEdge{1}{1}{1}$} (B)
          (B) edge [loop right] node {$\TEdge{1}{0}{1}$} (B)
          (B) edge [bend left]  node {$\TEdge{0}{1}{1}$} (A);
  \end{tikzpicture}
  \caption{This is the \eT for the Feedback XOR channel.}
  \label{fig:feedback_xor_channel}
\end{figure}

\section*{Acknowledgments}
\label{sec:acknowledgments}

We thank people for things.

\bibliography{chaos,ref}

\cleardoublepage

\appendix

\section{Appendix A}
\label{sec:appendix_a}

This is an appendix, if it is needed.

\end{document}
